\documentclass{article}
\usepackage[utf8]{inputenc}
\title{Converted PDF to LaTeX}
\author{Automated Script}
\date{\today}

\begin{document}

\maketitle

% Page 1
 pg. 1 Arduino: Build an IoT Environment Monitor   A synopsis report submission  for the first-year project of   Master of Technology in Computer Science and Engineering     By  Same er Dixit  (Roll No.:  230216008 )    UNDER THE SUP ERVISION OF   Dr. Rashi Agarwal   Associate Professor   SUBMITTED TO:   Dr. B B Sagar  (Associate Professor)   Dr. Shashwati Banerjea (Associate Professor)     DEPA RTMENT OF C OMPU TER SCIENCE AND ENGINEERING     Harcour t Butler Techni cal University , Kanpur Uttar Pradesh     

% Page 2
 pg. 2 Project Report : Department of Computer Science and Engineering   Arduino: Build an IoT Environment Monitor   1. Introduction   In today's rapidly evolving world, environmental monitoring plays a crucial role in ensuring  the well -being of our surroundings. This project focuses on building an Internet of Things  (IoT) environment monitor using Arduino. The device collects data from various sensors  and transmits it to the cloud for real -time tracking. This project aims to monitor key  environmental parameters such as temperature, humidity, air quality, and li ght intensity.   The scope of this project covers the use of Arduino as the hardware platform, interfacing it  with sensors, and leveraging IoT technology for data transmission and visualization.  IoT  environmental monitors play a significant role in real -time track ing of conditions . On the  other hand, the cloud computing space is filled with numerous cloud service providers  (CSPs) that offer varying levels of performance, pricing, and reliability  that we will be  discussing later in our report . Machine learning model s provide a powerful tool to rank  CSPs based on user requirements .  2. Moti vation   IoT-based environmental monitoring systems  have been explored in various applications,  from smart homes to industrial systems. Existing solutions primarily focus on temperature  and humidity monitoring, but recent advancements have incorporated air quality and light  sensors as well. This project bu ilds on these technologies, offering a cost -effective solution  with real -time monitoring capabilities.   Similarly, cloud service providers play a critical role in delivering infrastructure, platform,  and software services. A la rge body of research has focused on comparing CSPs based on  key performance indicators like cost, reliability, and scalability. Machine learning has  emerged as a key approach in evaluating CSPs due to its ability to process large datasets and  identify patt erns.   Similarly, cloud service providers play a critical role in delivering infrastructure, platform,  and software services. A la rge body of research has focused on comparing CSPs based on  key performance indicators like cost, reliability, and scalability. Machine learning has  emerged as a key approach in evaluating CSPs due to its ability to process large datasets and  identify patt erns.   3. Problem Statement   Cloud Service Pr oviders   There are numerous CSPs across the globe with different services, pricing models, and  infrastructures.  

% Page 3
 pg. 3 The list of major CSPs across the globe. As per Gartner's Magic Quadrant for Cloud  Infrastructure and Platform Services published in 2021 {Get this report: Gartner's Magic  Quadrant for Cloud Infrastructure and Platform Services} .  The top four CSPs are:   (1) Amazon Web Services (AWS),   (2) Microsoft Azure,   (3) Google Cloud Platform (GCP),   (4) IBM Cloud.   Once the top four vendors are identified, the primary goal is to choose the most appropriate  one among them for user's software req uirement specification.   Components and Tools   The components and tools used in this project are categorized into hardware and software:   Hardware   - Arduino Uno   - DHT11/DHT2 2 (Temperature and Humidity Sensor)   - MQ135 (Air Quality Sensor)   - Light Sensor (LDR)   - Wi-Fi Module (ESP8266/ESP32)   - Jumper Wires and Breadboard   Software   - Arduino IDE for coding   - ThingSpeak/Blynk for IoT platform   - APIs for data visualization   4. Abstract   The system consists of sensors connected to the Arduino board, which processes the data  and sends it to an IoT platform via the Wi -Fi module.   Block Diagram   The block diagram includes sensors (DHT11, MQ135, LDR), Arduino, Wi -Fi module, and the  cloud for  data storage and visualization.  

% Page 4
 pg. 4   Flowchart   1. Sensors collect environmental data.         

% Page 5
 pg. 5 2. Data is processed by Arduino.                     

% Page 6
 pg. 6 3. Wi -Fi module sends data to the cloud.                     

% Page 7
 pg. 7 4. The user accesses the data via an IoT dashboard.       5. Goals and Objectives with Circuit Design and Assembly   The circuit invol ves connecting the sensors to the respective pins of the Arduino board, and  setting up the Wi -Fi module for internet connectivity.   Power is supplied through a USB connection or a dedicated power source, and appropriate  jumper wires are used to establish co nnections between the components.   

% Page 8
 pg. 8   Software Implementation   The Arduino code initializes the sensors, collects data, and sends it to the IoT platform via  the Wi -Fi module. Below is a brief code snippet to illustrate how the temperature and  humidity sensor  (DHT11) is initialized.   // Code snippet example for DHT11 initialization   void setup () {    Serial.begin(9600);     dht.begin();   }  void loop() {     float humidity = dht.readHumidity();     float temperature = dht.readTemperature();   

% Page 9
 pg. 9   // Send data to cloud   }  The data collected from the sensors is sent to an IoT platform like ThingSpeak or Blynk. This  platform provides real -time monitoring and historical data visualization.   Steps to set up ThingSpeak/Blynk:   1. Create an account on the platf orm.   2. Set up a new channel for data logging.   3. Configure the Arduino code to send data to the platform using API keys.   6. Literature Surve y and Research Methodology   � Wide Detection Range : It can detect a range of harmful gases, making it suitable for  both industrial and domestic applications.   � Sensitivity : The sensor provides an analog output corresponding to the concentration of  gases, a llowing for precise monitoring.   � Low Cost : MQ135 is one of the more affordable gas sensors, making it popular for  DIY projects, IoT systems, and academic purposes.   � Durability : It has a long lifespan and works reliably in various environmental  conditio ns.  � Easy to Interface : It can easily be integrated with microcontrollers like Arduino,  Raspberry Pi, or other IoT platforms via its analog and digital pins.     Applicat ions:  � Air Quality Monitoring : Commonly used in IoT -based systems to monitor indoor air  quality, especially in homes, offices, and schoo ls.  � Industrial Safety Systems : Used to detect the presence of harmful gases in factories or  workplaces to ensure safety standards.   � Smart Homes : Integrated with home automation systems for real -time air quality  checks and triggering alarms or ventilation systems if unsafe gas levels are detected.   � Environmental Monitoring : Applied in environmental stations to detect pollution  levels in urban ar eas or near industrial zones.     

% Page 10
 pg. 10 IoT Cloud Integration   The data collected from the sensors is sent to an IoT platform like ThingSpeak or Blynk. This  platform provides real -time monitoring and historical data visualization.   Steps to set up ThingSpeak/Blynk:   1. Create an account on the platf orm.   2. Set up a new channel for data logging.   3. Configure the Arduino code to send data to the platform using API keys.   Testing and Results   The environment monitor was tested in different conditions to measure temperature,  humidity, air quality, and light levels. The following graphs show the data collected during  the testing phase, highlighting variations in environmental factors over time .    

% Page 11
 pg. 11 Applications and Future Scope   This project has applications in smart homes, industries, agricul ture, and environmental  research. Future work may include integrating additional sensors (e.g., CO2 sensors) and  leveraging machine learning for predictive environmental analysis.         

% Page 12
 pg. 12 Conclusion   In conclusion, this project successfully demonstrates the int egration of Arduino with IoT for  real -time environmental monitoring. It offers a cost -effective solution for tracking various  environmental parameters and provides significant opportunities for future improvements.   7. References   1. Arduino Documentation. (https://www.arduino.cc )  2. ThingSpeak API Documentation. ( https://thingspeak.com )  3. Blynk IoT Platform. ( https://blynk.io )  4. Gartner Says Worldwide IaaS Public Cloud Services Revenue Grew 16.2% in 2023.   https://www.gar tner.com/reviews/market/strategic -cloud -platform -services   5. How to choose a cloud service provider - AZURE :  https://azure.microsoft.com/en - in/resources/cloud -computing -dictionary/choosing -a-cloud -service -provider   6. Pooja Goyal, Sukhvinder Singh Deora, Cloud s ervice ranking with an integration of k - means algorithm and decision -making trail and evaluation laboratory approach.  Year - 2024 Available:   https://ijece.iaescore.com/index.php/IJECE/article/view/33869   7. International Journal of Research Publication and Rev iews :  https://ijrpr.com/uploads/V3ISSUE10/IJRPR7617.pdf   8. Indian  Institute of Embedded Systems on MQ1 35 sensor study:  https://iies.in/blog/what -is-the-mq-135 -sensor -and-how -does -it-work/      

\end{document}